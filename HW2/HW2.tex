\documentclass{report}
\usepackage{blindtext}
\usepackage[utf8]{inputenc}
\usepackage{titlesec}
\usepackage{mathtools}
\newcommand{\me}[1]{
\begin{math}
#1
\end{math}
}
\title{Home Work 2}
\author{Francisco J. Díaz Riollano \\ Student ID: 802-15-2172 }


\begin{document}

\maketitle
 %%%%%%%%%%%%%%%%%%% First Question%%%%%%%%%%%%%%%%%%%%%%%
\paragraph{\Large{Question 1.4.3\\ \\}}
 Is it the case that for each \me{i \in \mathbb{N},L^i \cap L^{i+1} = \o} ? If this indeed the case, prove it. Otherwise provide a counterexample.\\
 \\ \\
%%%%%%%%%%%%%%%%%%% First Question%%%%%%%%%%%%%%%%%%%%%%%%
Let L be a formal language over an alphabet \me{\Sigma}.\\  Such that \me{L = \{l:} l is a string over \me{\Sigma \}}
\\ \\
Proof:
We first demonstrate by induction that \me{\forall a \in L^n,|a| = n}. \\ \\
Then the proof will go as follows:
\\ \\
\textbf{Base Case:} \\ \\
For \me{n=1}. Since L is a language over an alphabet \me{\Sigma} each \me{a \in L} is a symbol, and therefore \me{|a|=1.}
\\ \\
\begin{flushleft}
	\textbf{Inductive Hypothesis:} \me{\forall a \in L^n, |a|=n}. Then,
	\me{ L^{n+1} = L^nL= \{w:w =xy| x \in L^n \land y \in L\}} by property of concatenation.\\
	Since \me{|xy| = |x| + |y|}, using the inductive hypothesis we get me{|xy| = n +1}.\\
	By proving that every element in \me{L^n} cannot possibly be \me{L^{n+1}} and viceversa, the elements of each of these formal languages have different string length, we can conclude:
	\\ 
	\me{i \in \mathbb{N},L^i \cap L^{i+1} = \o} 




\end{flushleft}



 %%%%%%%%%%%%%%%%%%% Second Question%%%%%%%%%%%%%%%%%%%%%%%
 
%%%%%%%%%%%%%%%%%%% Second Question%%%%%%%%%%%%%%%%%%%%%%%

\end{document}
