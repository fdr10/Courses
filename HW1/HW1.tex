\documentclass{report}
\usepackage{blindtext}
\usepackage[utf8]{inputenc}
\usepackage{titlesec}
\usepackage{mathtools}
\newcommand{\me}[1]{
\begin{math}
#1
\end{math}
}
\title{Home Work 1}
\author{Francisco J. Díaz Riollano \\ Student ID: 802-15-2172 }


\begin{document}

\maketitle
 %%%%%%%%%%%%%%%%%%% First Question%%%%%%%%%%%%%%%%%%%%%%%
\paragraph{\Large{Question 1.4.1\\ \\}}
 Can you find examples of unions of strings and alphabets that are not string alphabets?\\
 \\
 Let  \me{\Sigma_0 = \{0,1\}} and let  \me{\Sigma_1 = \{00,11\}}. Then \me{\Sigma = \Sigma _0 \cup \Sigma_1 } = \me{\{0,1,00,11\}.} Then \me{\rho(\omega) = (0,0,1,1)} or \me{\rho(\omega) = (0,0,11)} or \me{\rho(\omega) = (00,11)} or \me{\rho(\omega) = (00,1,1)}.
 \\ 
 In this example the string mapping is not well defined, therefore \me{\Sigma_0 \cup \Sigma_1} is not a \textit{string alphabet}
 \\ \\ \\
 Another example would be: \\
 Let \me{\Sigma_2 = \{a,b\}} and let \me{\Sigma_3 = \{baa\}} \\
 Let \me{\Sigma = \Sigma_2 \cup \Sigma_3} = \me{\{a,b,baa\}}\\
 Let \me{\omega = baa} which is a string of the alleged alphabet.
 Then \me{\rho(\omega) = (b,a,a)} or  \me{\rho(\omega) = (baa)}. Thus, this other example is not a \textit{string alphabet}.
 %%%%%%%%%%%%%%%%%%% First Question%%%%%%%%%%%%%%%%%%%%%%%%
 
 
 
 %%%%%%%%%%%%%%%%%%% Second Question%%%%%%%%%%%%%%%%%%%%%%%
 
\paragraph{\Large{Question 1.4.2\\ \\}}
Alice said that any finite union of string of the same length is a string alphabet. Is Alice right?
\\ \\
Proof: \\\\
Let P(x) =: "x is a string alphabet"\\
Let \me{\Sigma_0 = \{ a1,a2,\dots,ak\}} and let  \me{\Sigma_1 = \{ b1,b2,\dots,bn\}} 
\\ \\
Let \me{\Sigma = \{ \omega \mid (\omega = \Sigma_0 \cup \Sigma_1) \land ((\forall a \in \Sigma_0)(\forall b \in \Sigma_1 )|a| = |b|) \}}
\\ 
\\
\me{(\forall \omega \in \Sigma)P(\omega)}
\\ 
\\
\\
\me{(\forall \omega_0 \in \Sigma_0)(\forall \omega_1 \in \Sigma_1)} \me{ \omega_0 \notin \Sigma \iff (\omega_0 = \omega1) \land (\omega_1 \in \Sigma)}, by property of sets. Thus \me{\omega \in \Sigma} is unique and of the same length. Since every element in \me{\Sigma},when given a tuple \me{\rho(\omega ^*)} = T, then \me{(\forall t \in T)} will have a single mapping to the \textit{string alphabet}\me{\Sigma}.
\\
\\
\begin{flushright}
Q.E.D
\end{flushright}

 %%%%%%%%%%%%%%%%%%% Second Question%%%%%%%%%%%%%%%%%%%%%%%

\end{document}