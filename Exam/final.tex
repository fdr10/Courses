\documentclass{report}
\usepackage{blindtext}
\usepackage{amssymb}
\usepackage[mathscr]{euscript}
\usepackage[utf8]{inputenc}
\usepackage{algorithm}
\usepackage[noend]{algpseudocode}
\usepackage{titlesec}
\usepackage{mathtools}
\usepackage{tikz} %% Package for drawing finite state machines in LaTEX
\usetikzlibrary{automata, positioning,arrows}
\newcommand{\me}[1]{
\begin{math}
#1
\end{math}
}
\title{Exam 2}
\author{Francisco J. Díaz Riollano \\ Student ID: 802-15-2172 }


\begin{document}

\maketitle
 %%%%%%%%%%%%%%%%%%% First Question%%%%%%%%%%%%%%%%%%%%%%%
1) (25 pts.) The argument to prove that the complement of a regular language is also a regular language is: \\
\textit{“Since L is regular, there is Deterministic Finite State Automaton D that recognizes L. But then, the Deterministic Finite State Automaton $\bar{D}$ obtained interchanging accept and non-accept states of D, recognizes $\bar{L} $, the complement of L.”}  \\ \\ Is the argument valid when Deterministic FSA is replaced with Non-deterministic FSA and no transformation from Non-deterministic to Deterministic FSA is invoked?\\[0.2in] 


2) (25 pts). Let $L = \{< a,b,c,p >: $ a,b,c and p  are integers $a^p \equiv c(mod p) \}$. Demonstrate that $L$ is in $P$.\\[0.2in] 




3) (25 pts) Consider the problem $ L = \{<T,w>: $  $T$ is a Turing Machine and $w$ a fixed string such that $T$ enters each of its states on input $w$    $\}$\\[0.2in] 




4) (25 pts) Consider the problem $L=\{ <G,w>:G$ is a context free grammar and $w$ a string, such that $w\in L(G)\}$. Is $L$ decidable? Provide a formal answer.







 %%%%%%%%%%%%%%%%%%% Second Question%%%%%%%%%%%%%%%%%%%%%%%

\end{document}