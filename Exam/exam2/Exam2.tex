\documentclass{report}
\usepackage{blindtext}
\usepackage{amssymb}
\usepackage[mathscr]{euscript}
\usepackage[utf8]{inputenc}
\usepackage{algorithm}
\usepackage[noend]{algpseudocode}
\usepackage{titlesec}
\usepackage{mathtools}
\usepackage{tikz} %% Package for drawing finite state machines in LaTEX
\usetikzlibrary{automata, positioning,arrows}
\newcommand{\me}[1]{
\begin{math}
#1
\end{math}
}
\title{Exam 2}
\author{Francisco J. Díaz Riollano \\ Student ID: 802-15-2172 }


\begin{document}

\maketitle
 %%%%%%%%%%%%%%%%%%% First Question%%%%%%%%%%%%%%%%%%%%%%%
\paragraph{\Large{Question 1\\ \\}}
1. (20 points) Demonstrate that a deterministic Turing machine can be simulated with a pushdown automaton with two stacks.  \\ \\
Definitions: \\
Turing Machine, \me{TM =(Q,\Sigma, \Gamma, \delta, q_0,q_{accept}, q_{reject}). } \\
1) Q is the set of states \\
2)\me{\Sigma} is the input alphabet not containing the blank symbol. \\
3) \me{\Gamma} is the tape alphabet, where $\sqcup \in \Gamma $ and $\Sigma \subseteq \Gamma$ \\
4) \me{\delta: Q \times \Gamma \to Q \times \Gamma \times  \{L,R\}}\\
5) $q_0$ is the start state\\
6) $q_{accept}$ is the accept state \\
7) $q_{reject}$ is the reject state, where $q_{reject} \neq q_{accept}$
\\ \\
2-Stack Pushdown Automata, \me{P =(Q,\Sigma, \Gamma_{\varepsilon L},\Gamma_{\varepsilon R}, \delta, q_0,F)} \\
1) Q is the set of states \\
2)\me{\Sigma} is the input alphabet not containing the blank symbol. \\
3) \me{\Gamma} is the stack alphabet \\
4) \me{\delta: Q \times \Sigma_{\varepsilon} \times \Gamma_{\varepsilon R} \times \Gamma_{\varepsilon L}  \to } $\mathcal{P}(Q \times \Gamma_{\varepsilon L}\times\Gamma_{\varepsilon R})$, where $\Gamma_{\varepsilon L} $ is the the left stack with    $\varepsilon$ and $\Gamma_{\varepsilon R}$ is the right stack with $\varepsilon$ \\
5) $q_0 \in Q$ is the start state\\
6) $F \subseteq Q$ is the set of accept states. \\ \\ 


The idea is to show that the transition function in a TM; \me{\delta: Q \times \Gamma \to Q \times \Gamma \times  \{L,R\}} is equivalent to the transition function of a 2-Stack Pushdown Automata; \me{\delta: Q \times \Sigma_{\varepsilon} \times \Gamma_{\varepsilon R} \times \Gamma_{\varepsilon L}  \to } $\mathcal{P}(Q \times \Gamma_{\varepsilon L}\times \Gamma_{\varepsilon R})$. \\

We define the transition function of the pushdown automata on which the movement to the left of the TM, say \me{\delta(q_1,0) = (q_2,\times,L)} for example, to be equivalent to pushing to the right stack \me{\Gamma_{\varepsilon R} } and popping on the left stack stack \me{\Gamma_{\varepsilon R} }. Vice versa, moving to the right in the TM would be equivalent to pushing the symbol it read to the left stack and performing a pop from the right stack.  Since a turing machine cannot read an empty string from the tape alphabet it does however, have $\sqcup$. Thus operations involving $\sqcup$ in the TM would be equivalent to operations involving $\varepsilon$ on the 2-Stack Pushdown Automata. \\ \\
As for the starting, accepting, rejecting states. \\ \\
In the 2-Stack Push Down Automata $q_0$ would be the same as $q_0 $ in the TM. \\
$F$, the subset of accepting states in the 2-Stack Pushdown Automata, must contain $q_{accept}$ state of the TM.\\ \\
For the rejecting state $q_{reject}$ of the TM must be a state in $Q - F$ of the 2-Stack Push Down Automata in order to equivalent.
 %%%%%%%%%%%%%%%%%%% First Question%%%%%%%%%%%%%%%%%%%%%%%%
 
 
 
 %%%%%%%%%%%%%%%%%%% Second Question%%%%%%%%%%%%%%%%%%%%%%%
 \newpage
\paragraph{\Large{Question 2\\ \\}}
2. (a) (10 points) Define formally the concepts of decidable language and Turing-recognizable language, and show that a decidable language is Turing-recognizable but not necessarily the other way around.  \\  \\
1) Decidable languages:\\  A language $L$ over $\Sigma$ is said to be to be decidable if there exist a Turing Machine T, such that \me{L = L(T) }. Such a language is said to be decidable, if on input $w$, the machine $accepts$ if $x \in L$ and the machine $rejects$ otherwise.  Often such a machine is called a decider of the language $L$.  \\ \\
2) Turing recognizable language: \\A language $L$ over $\Sigma$ is said to be recognizable 	if there exists a Turing Machine T, \me{L =L(T)} but T is not necessarily a decider. Meaning such a machine may not necessarily tell whether on input $x \in L$ will $accept$ or 
$reject$. \\ \\3) Proof: \\ We have two directions to prove. The first will consist of proving that if a language is decidable it is also recognizable. The other proof will consist of proving that if a language is recognizable does not necessarily imply the language is decidable. The later could be done by proving the existence of language that is recognizable but is not decidable. \\ \\
a) We will show this, by proving there exist such a language of a decider, $A_{TM}$, that decides whether a Turing Machine is Turing recognizable. \\ \\
$U =$  ``On  input $<M,w>$, where $M$ is a TM and $w$ is a string: \\

   \indent 1) Simulate $M$ on $w$.\\
   \indent 2) If $M$ ever enters its accept state, $accept$; if M ever enters its reject state, \indent \indent$reject$" \\ \\
b) The other proof will consist of showing that if language is recognizable, it is not necessarily a decidable language. \\ 
\indent We assume $A_{TM}$ is decidable and obtain a contradiction. \\ 
Suppose $H$ is a decider for $A_{TM}$. Namely \me{H(<M,w>)}, $accepts$ if M accepts $w$, $rejects$ if $M$ does not accept $w$. Now we will construct a new machine $D$ which determines what $M$ does when given its own description $M$. \\
\indent \me{D =} `` On input $<M>$, where M is a TM: \\
\indent \indent 1. Run H on input $<M,<M>>$. \\
\indent \indent  2. Output the opposite of what H outputs; that is, if H accepts $reject$, \indent \indent  if H accepts $reject$.\\ Now run D on itself. \\ \\

\me{D(<D>)}: $accept$ if D does not accept $<D>$, \\ \indent \indent \indent \indent \indent$reject$ if D accepts $<D>$. \\
No matter what D does, it is forced to do the opposite, which is a contradiction. Since we have found a language that is recognizable but not decidable we can conclude the statement as true. 

\newpage

(b) (30 points)  Let I$NF_A =  \{<A>: A $ A is a finite state automaton and \me{L(A)} is infinite $\} $. Is $INF_A$ decidable? Answer with a demonstration or a counterexample. \\


We want prove that given a an Automata, $A$ has an a language $L(A)$, which is infinite. \\ We cannot test all \me{w \in \Sigma^*}. However, we have the pumping lemma to help us. A language that satisfies the pumping lemma of finite state machines is of the form $xy^iz$, where \me{i >0} and \me{p=|y|>0}. If we can find a decider such that decides on languages that have such a string with a least length $p$ then, our language is infinite by the pumping lemma. \\ \\
 Pseudocode:
 
 M = On input A \\
  Let $p$ be the number of states in A (the cardinality of the set of states \me{|Q|})
  Let $\Sigma$ be the alphabet of the automata. \\
  Let $k \in \mathbb{N}, k>1$ (for simplicity choose k =2).\\
  For each string \me{w \in \Sigma^*} such that \me{p \leq |w| \leq k\times p}\\
  \indent Run $M_{DFA}(<A,w>)$, where $M_{DFA}$ decides whether $A$ accepts $w$ or not \\
  \indent If $M_{DFA}$ accepts\\ 
  \indent \indent $Accept$ \\
  End For Loop \\
  $Reject$



\newpage
3. (a) (10 points) Define formally the concept of computable function. \\\\
\\
A computable function is a function is one whose values can be computed with a Turing Machine.
A function \me{f:\Sigma^* \to \Sigma^*} is said to be computable, if there is a Turing Machine M which on input $w \in \Sigma^*$, will halt with $f(w)$ on the machines tape. The machine $w$ may accept or reject $w$. Also computable functions may not be described with a closed formula. This will be left to reduction which may or may not happen. \\ \\
(b) (30 points ) Prove or disprove with a counterexample that the reductions of \\
\indent (a) $ AT_M$ to $HALT $\\
\indent (b) $ E_{TM}$ to $EQ_{TM} $ are both computable.

 %%%%%%%%%%%%%%%%%%% Second Question%%%%%%%%%%%%%%%%%%%%%%%

\end{document}