\documentclass{report}
\usepackage{blindtext}
\usepackage{amssymb}
\usepackage[mathscr]{euscript}
\usepackage[utf8]{inputenc}
\usepackage{algorithm}
\usepackage[noend]{algpseudocode}
\usepackage{titlesec}
\usepackage{mathtools}
\usepackage{tikz} %% Package for drawing finite state machines in LaTEX
\usetikzlibrary{automata, positioning,arrows}
\newcommand{\me}[1]{
\begin{math}
#1
\end{math}
}
\title{Final Exam}
\author{Francisco  Diaz  }


\begin{document}

\maketitle
 %%%%%%%%%%%%%%%%%%% First Question%%%%%%%%%%%%%%%%%%%%%%%
1) (25 pts.) The argument to prove that the complement of a regular language is also a regular language is: \\ \\
\textit{“Since L is regular, there is Deterministic Finite State Automaton D that recognizes L. But then, the Deterministic Finite State Automaton $\bar{D}$ obtained interchanging accept and non-accept states of D, recognizes $\bar{L} $, the complement of L.”}  \\ \\ Is the argument valid when Deterministic FSA is replaced with Non-deterministic FSA and no transformation from Non-deterministic to Deterministic FSA is invoked?\\[0.2in] 



Proof: \\ \\ 

Let $A$ be a non-deterministic finite state machine $A = (Q, \Sigma, \delta, q_0, F)$  and $L = L(A)$ be the language that the finite state machine recognizes. We want to prove the existance of another non-deterministic finite state machine that recognizes $\bar L$. The idea is to build a machine $A'$ which accepts when $A$ rejects. Let $A' = (Q,\Sigma, \delta,q_0, Q-F)$ , where $Q-F$ is the set of states that are in $Q$ but not in $F$. Since $F$ contains the set of accepting states, then any input string $w$ that does not land on one these states, the machine $A$, is said to reject and by the construction above $A'$ will thus accept. Thus $A'$ will accept anything input string of the form $w \in \bar L$, where  $\bar L = \Sigma^*- L$. This proves that non-deterministic machines can recognize the complements of regular languages and that the complement of regular languages are also regular languages.    \\ \\


2) (25 pts). Let $L = \{< a,b,c,p >: $ a,b,c and p  are integers $a^b \equiv c \mod p \}$. Demonstrate that $L$ is in $P$.\\[0.2in] 
By defintion $a^b \equiv c \mod p $ is the same as $ p | a^b - c$, which read $p$ divides $a^b - c$. As a direct consequence of the above $a^b \equiv c \mod p $ iff $a^b \mod  p = c \mod p$. Roughly speaking, congruence $\iff$ same remainder. We could design a Turing Machine $M$ that accepts if and only if $a^b$ and $c$ have the same remainder when taken the modulus $p$  \\ \\
// E is a subroutine \\ \\ 
E = `` On input $<x,y>$ where $x,y$ are integers in binary\\
1) Repeat until $x < y$ \\
	\indent \indent  Assign $x = \lceil x/y\rceil $ \\
2) Output $x$" \\ \\ \\ M = `` On input $<a,b,c,p>$ , where $a,b,c,p$ are integers in binary \\
1) Run E on $< a^b, p>$ \\
 \indent \indent Assign the output to $x$ \\
 2) Run E on $<c,p>$  \\
 \indent \indent Assign the output to $y$ \\
 3) If $x == y$, $ accept$ \\
 4) $reject$ \\ \\
 
 
 The complexity of the division in subroutine E  is $O(n^2)$, where $n$ is an n-digit number. Since subroutine E will do at most $k$ iterations. In algorithm M we perform two of these operations, $2O(n^2k)$ and a comparison, $O(1)$ so the algorithm is of the order $O(n^2k)$, we can guarantee that $k$ will be no longer than $n$ itself (this is due to asymptotic nature of division) therefore: \\ \\
 $L = \{ <M,a,b,c,p> | M$ is a TM that checks $a^p \equiv c \mod p\} $ belongs to the class $P$. \\ \\ \\ \\
 
 
 


3) (25 pts) Consider the problem $ L = \{<T,w>: $  $T$ is a Turing Machine and $w$ a fixed string such that $T$ enters each of its states on input $w$    $\}$. Is $L$ undecidable? Provide a formal answer.\\[0.1in] 

We will show that $L$ is undecidable by a reduction of $A_{TM}$ is reducible to L, where $A_{TM} = \{<M,w> | M $ is a $TM$ and $M$ accepts $w\} $. \\  Suppose that L is decidable and that TM $R$ decides it. Since $R$ solves $L$, we can use $R$ to check if $w$ visits all of the states to decide $A_{TM}$. Below, I will construct a TM $S$ that ``decides" $A_{TM}$ by using the decider $R$ for $L$ as a subroutine. \\ \\S = `` On input $<T,w>$, where $T$ is a $TM$.\\
1) Run TM $R$ on input $<T,w>$\\
2) If $R$ accepts, then $accept$, If $R$ rejects, $reject$." \\ \\
However, since we know $A_{TM}$ is undecidable, there cannot exist a TM that decides L. \newpage





4) (25 pts) Consider the problem $L=\{ <G,w>:G$ is a context free grammar and $w$ a string, such that $w\in L(G)\}$. Is $L$ decidable? Provide a formal answer. \\ \\

Any context-free language is decidable.  \\ \\ 
Proof: \\ \\ 
Let $L$ be a context-free language and $G$ be the context-free grammar that generates $L$, then for each $w\in \Sigma^* , w \in L   $ if and only if $S(<G,w>)$ accepts, thus $L$ is decided by $S(<G,>)$, where $S$ is the following algorithm:

\textbf{procedure} S (On input $<G,w>)$ \\
1) Transform $G$ into Chomsky normal form \\
2) Compute $n \leftarrow |w| $ \\
3) Enumerate all $2n-1$ steps generated by $G$ \\
4) for each $2n-1$ step derivations. \\
\indent if w is generated, \\
\indent \indent $accept$ \\
5) $reject$








 %%%%%%%%%%%%%%%%%%% Second Question%%%%%%%%%%%%%%%%%%%%%%%

\end{document}
